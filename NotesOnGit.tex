\documentclass[12pt]{article}
\usepackage{amsthm}
\usepackage{amsmath}
\usepackage{amssymb}
\usepackage{mathtools}
\usepackage{blindtext}
\usepackage[utf8]{inputenc}
\usepackage{enumerate}
\usepackage{hyperref}
\usepackage{listings}
\usepackage{bm}
\usepackage{enumitem}
\usepackage{multicol}
\usepackage[margin=1.0in]{geometry}
\usepackage[export]{adjustbox}
\usepackage[makeroom]{cancel}

\usepackage{setspace}
\onehalfspacing

\newtheorem{thm}{Theorem}
\newtheorem{lem}[thm]{Lemma}
\newtheorem{defn}[thm]{Definition}
\newtheorem{eg}{Example}

\newcommand{\C}{\mathbb{C}}
\newcommand{\N}{\mathbb{N}}
\newcommand{\Z}{\mathbb{Z}}
\newcommand{\R}{\mathbb{R}}
\newcommand{\bigo}{\mathcal{O}}
\newcommand{\X}{\mathcal{X}}

\title{Building Software in Teams}
\author{Cassidy F. Krause}
\date{July 16, 2018}
\begin{document}
\maketitle

\section{Day 1 - Morning session.}
\subsection{Setting up Git and Github}
\begin{itemize}
\item Later today, read documentation about integration of Matlab and Git.
\item Git commands in command line: single dash for a letter, and double for a word.
    \subitem Eg: git config --global
\item To synchronize end of line character across different platforms:
    \subitem Windows: git config --global core.autofclf true
    \subitem Mac/Linux: git config --global core.autofclf input
\item Automatic text editor in GitBash is vi
\end{itemize}
\subsection{Intro to Git and GitKraken}
\begin{itemize}
\item Git is a distributed version control system. It is different from Github, which works as a sort of server for Git.
\end{itemize}

Suppose we have two different versions of the same code. Version 1, the original, has:
\begin{align*}
  &foo = 1;\\
  &bar = 2;
\end{align*} while Version 2 has an additional variable:
\begin{align*}
  &foo = 1;\\
  &bar = 2;\\
  &baz = 3;
\end{align*}
Each of these versions has a label, and we can create a directed graph of the changes between them.
\includegraphics[width = \textwidth]{BranchDiagram.png}

Each of the circles in the diagram would have its own label, and the pointer ``head'' would point to the current working directory. Git is a directed graph that saves the change sets. That is, each snapshot saves the changes, not the file itself. You can also name the branches and add tags for better organization.

The typical workflow is set up as:
\begin{itemize}
  \item Add branches to address bugs or add features
  \item Merge after separate issues have been addressed
  \item Use tags for different versions
\end{itemize}

This sort of workflow is much easier to see in the GUI (Eg., GitKraken) rather than the command line; however, the command line can do everything that GitKraken can do and is generally more powerful.

The general rule of thumb is to do version control on inputs (not outputs). For example, for this project, I am just tracking changes in the *.tex file, and not in the auxiliary or *.pdf files. Also, don't do version control with data. To ignore these auxiliary files, add their extensions to the gitignore file.

\emph{Now we continue to edit the branch, while the master has something else going on.}

Changing the master branch while we also have a different branch going.

\end{document}
